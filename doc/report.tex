\documentclass[12pt, a4paper, oneside]{article}
\usepackage{graphicx}
\usepackage{hyperref}
\usepackage{listings}
\usepackage{placeins}
\usepackage{amsmath}
\usepackage{amsfonts}
\usepackage{bussproofs}
\usepackage{xparse}

\usepackage[toc,page]{appendix}

\newcommand{\stepsto}{\Downarrow}
\newcommand{\hoare}[3]{
	 $\left\langle #1 \right\rangle   #2 \left\langle #3 \right\rangle$
}
% font size could be 10pt (default), 11pt or 12 pt
% paper size coulde be letterpaper (default), legalpaper, executivepaper,
% a4paper, a5paper or b5paper
% side coulde be oneside (default) or twoside 
% columns coulde be onecolumn (default) or twocolumn
% graphics coulde be final (default) or draft 
%
% titlepage coulde be notitlepage (default) or titlepage which 
% makes an extra page for title 
% 
% paper alignment coulde be portrait (default) or landscape 
%
% equations coulde be 
%   default number of the equation on the rigth and equation centered 
%   leqno number on the left and equation centered 
%   fleqn number on the rigth and  equation on the left side
%   

\newcommand\doubleplus{+\kern-1.3ex+\kern0.8ex}
\newcommand\mdoubleplus{\ensuremath{\mathbin{+\mkern-10mu+}}}

\newcommand\tassig{:=}

\newcommand{\sproof}{
  \scriptsize
  \begin{center}
  \begin{prooftree}
  \def\defaultHypSeparation{\hskip .1in}
% whats that doing? ->  \def\fCenter{\models}
}
\newcommand{\eproof}{
  \end{prooftree}
  \end{center}
  \normalsize
}


\newcommand{\nsproof}{
  \scriptsize
  \def\defaultHypSeparation{\hskip .1in}
% whats that doing? ->  \def\fCenter{\models}
}
\newcommand{\neproof}{
  \DisplayProof
  \normalsize
}

\DeclareDocumentCommand{\sstepe}{m m m m} {%
  \ensuremath{{#1} \vdash \text{{#3}} \; \rightarrow \; {#2} \vdash \text{{#4}}}%
}
\DeclareDocumentCommand{\sstep}{O{} O{} m m} {%
  \sstepe{\Phi_{#1}}{\Phi_{#2}}{#3}{#4}%
}



\title{TPT Project - Heap Semantics}
\author{Marco Vassena  \\
    4110161 \\
    \and
    Philipp Hausmann \\
    4003373 \\
    \and
    Ondra Pelech \\
    F131636 \\
    }

\date{\today}
\begin{document}



\maketitle

\tableofcontents


\section{Research question}
Defining a formal semantics for a small arithmetic language is straightforward. In this project, we aimed to enrich such a
minimal programming language with a mutable heap, defining and formalizing its semantics. Adding mutable, global state makes
the language impure, which  poses a number of problems and issues, such as threading the heap through the evaluation of a term
and dealing with invalid references to the heap.

We managed to solve these problems and reached our goals. Furthermore we have successfully embedded hoare logic in the language,
allowing to reason about the correctness of a program written in this minimal programming language.

\subsection{Overview}
In section~\ref{sec:method} we show different alternative approaches to solve the problem and their restrictions. 
In section \ref{sec:cont} we define the formal semantics in Agda for one such approach.
Section~\ref{sec:metatheories} lists the theoretical results proved and in~\ref{sec:hoare} the embedding of hoare logic is explained.
Finally in~\ref{sec:conclusion} we draw our conclusions about the project, compare it to the languages studied during the course and propose few ideas for future work.

\section{Research Method}
\label{sec:method}
Designing a suitable language to study the semantics of a mutable heap is not inherently complex, but influences greatly
the ability to prove things. Therefor the choice of language has to be done careful, and we identified two
major aspects in defining such a language.

\subsection{Totality}
The first aspect is whether the language is total or not. We will call a language total if all heap accesses are always guaranteed
to succeed. A language where any heap access might fail or diverge at runtime will be called non-total.

The available heap operations include new allocations, dereferencing and assignment. An example of the
terms for the non-total version is given in figure \ref{lst:part-term}.

\begin{figure}
{\scriptsize
\begin{lstlisting}
data Term : Type -> Set where
  new  : forall {ty} -> Term ty -> Term (Ref ty)
  !_   : forall {ty} -> Term (Ref ty) -> Term ty
  _:=_ : forall {ty} -> Term (Ref ty) -> Term ty -> Term ty
  ref  : forall {ty} -> Nat -> Term (Ref ty)
\end{lstlisting}}
\caption{Heap related terms for the partial language.}
\label{lst:part-term}
\end{figure}

\paragraph{Total version}
Having a total language on one hand simplifies proofs in that it reduces the number of possible evaluations.
On the other hand, it requires that we can guarantee the totality of the language in the proofs. This turned
out to be rather difficult.

This proof can be encoded in Agda by adding a heap description object which we call shape. The shape
describes the cells of the heap and can be used to guarantee the success of a heap access. The design
is very similar to the Context/Environment design used in the Simply Typed $\lambda$-Calculus
during the TPT lectures.

While this provides a promising foundation to implement such a language in Agda, the details depend
heavily on the design choices and restrictions present in the language.

First, a language may allow storing references to be stored on the heap. This leads to almost circular
dependencies, because the heap elements may now indirectly depend on the heap. Because of the different
versions of the heap involved, it is unlikely that a real circular dependency exists. Nonetheless,
it complicates the proof that the language is total.

Second, references are normally regarded as values. To be able to relate the references with the
heap present during evaluation, the references need to be indexed by the shape. This in turn requires
that values must be indexed by the shape, which is rather counterintuitive and verbose.

Other aspects may interfere with the definition of a total language in this setting as well.

Due to the complexity involved in designing such a language, we do not have a finished solution for this approach.
While we are not aware of any insurmountable problem, defining a suitable encoding in Agda
and proving properties of its semantics is non-trivial.


\paragraph{Partial version}
Using a non-total language on the other hand is relieved from the burden of guaranteeing totality.
Instead, the concept of failure now needs to be encoded either by defining a partial semantics
or by lifting the partiality to the language itself.

A first attempt to define a partial semantics turned out to be fairly awkward and non-idiomatic. Proofs are limited only to the succeeding cases, which leads to verbose and error-prone proofs without gaining any benefit. 

Lifting the partiality to the language itself makes the semantics total again.
Handling errors appropriately is then a concern of the final user, which mimics the semantics of most widely used programming languages like C, Java, etc. in this regard.
The terms for the partial heap-approach can be seen in figure \ref{lst:part-term}.



\subsection{Host Language}
The second important choice is what other features apart from the heap itself the language will have. We will call this the \emph{Host Language} in the remainder of this report.

\paragraph{BoolNat}
The BoolNat language is a minimal language containing only boolean expressions and natural numbers. It already allows studying
the semantics of a heap while keeping the complexity at a minimum.

Our result is based upon a slightly extended version of this language, which will be explaind in section \ref{sec:boolnat}.

\paragraph{Simply Typed $\lambda$-Calculus}
Our first approach was to add a mutable heap to the Simply Typed $\lambda$-Calculus using a total semantics. We used the
semantics discussed during the TPT lectures as basis for this.

The combined complexity of the Simply Typed $\lambda$-Calculus itself together with the totality requirement prevented that a solution could be found in a useful timeframe.

How a partial semantics would work together with Simply Typed $\lambda$-Calculus has not been looked at in this project, but we expect
that our work on partial semantics for the simpler BoolNat language described in the remainder of this report provides a good
foundation for further research in that direction.



\section{Partial BoolNat language}
\label{sec:boolnat}
Due to the problems encountered with the Simply Typed $\lambda$-Calculus and the totality requirement, we will focus on a partial BoolNat
language in the following chapters. To be more precise, it is a version of the BoolNat language extended with a try-catch construct and 
sequencing. The terms of the language can be seen in figure \ref{lst:part-boolnat-term}.
The big-step semantics of the heap can be found in figure \ref{fig:heap-big}. The ":=" symbol is used in this report for the assignment
operation for better readability. In the
Agda source code, assignment is represented by "\textless--".

The full small- and big-step semantics can be found in appendix \ref{app:small} and \ref{app:big}.

\begin{figure}
{\scriptsize
\begin{lstlisting}
data Term : Type -> Set where
  true          : Term Boolean
  false         : Term Boolean
  error         : forall {ty} -> Term ty 
  num           : Nat -> Term Natural
  iszero        : Term Natural -> Term Boolean
  if_then_else_ : forall {ty} -> (cond  : Term Boolean)
                              -> (tcase : Term ty)
                              -> (fcase : Term ty)
                              -> Term ty
  new           : forall {ty} -> Term ty -> Term (Ref ty)
  !_            : forall {ty} -> Term (Ref ty) -> Term ty
  _:=_          : forall {ty} -> Term (Ref ty) -> Term ty -> Term ty
  ref           : forall {ty} -> Nat -> Term (Ref ty)
  try_catch_    : forall {ty} -> Term ty -> Term ty -> Term ty
  _>>_          : forall {ty1 ty2} -> Term ty1 -> Term ty2 -> Term ty2
\end{lstlisting}}
\caption{The terms of the partial BoolNat language.}
\label{lst:part-boolnat-term}
\end{figure}

\begin{figure}
\nsproof
\RightLabel{New}
\AxiomC{$\Phi_1 \vdash$ t ~$\stepsto$~ $\Phi_2 \vdash$ v}
\AxiomC{}
\BinaryInfC{$\Phi_1 \vdash$ \textbf{new} t ~$\stepsto$~ $\Phi_2 \doubleplus$ [v] $\vdash$ \emph{vref} (\underline{length} $\Phi_2$)}
\neproof
\hfill
\nsproof
\RightLabel{Deref}
\AxiomC{$\Phi_1 \vdash$ t ~$\stepsto$~ $\Phi_2 \vdash$ \emph{vref} r}
\AxiomC{}
\BinaryInfC{$\Phi_1 \vdash$ \textbf{!} t ~$\stepsto$~ $\Phi_2 \vdash$ \underline{lookup} r $\Phi_2$}
\neproof
\\\\
\nsproof
\RightLabel{DerefErr}
\AxiomC{$\Phi_1 \vdash$ t ~$\stepsto$~ $\Phi_2 \vdash$ \emph{verror}}
\AxiomC{}
\BinaryInfC{$\Phi_1 \vdash$ \textbf{!} t ~$\stepsto$~ $\Phi_2 \vdash$ \emph{verror}}
\neproof
\hfill
\nsproof
\RightLabel{Ass}
\AxiomC{\textbf{Elem} $\Phi_3$ r ~~ $\Phi_1 \vdash$ t$_1$ ~$\stepsto$~ $\Phi_2 \vdash$ \emph{vref} r}
\AxiomC{$\Phi_2 \vdash$ t$_2$ ~$\stepsto$~ $\Phi_3 \vdash$ v}
\BinaryInfC{$\Phi_1 \vdash$ t$_1$ \textbf{\tassig} t$_2$ ~$\stepsto$~ \underline{replace} $\Phi_3$ r v $\vdash$ v}
\neproof
\\\\
\nsproof
\RightLabel{AssOob}
\AxiomC{$\neg$ \textbf{Elem} $\Phi_3$ r}
\AxiomC{$\Phi_1 \vdash$ t$_1$ ~$\stepsto$~ $\Phi_2 \vdash$ \emph{vref} r}
\BinaryInfC{$\Phi_1 \vdash$ t$_1$ \textbf{\tassig} t$_2$ ~$\stepsto$~ $\Phi_2$ $\vdash$ \emph{verror}}
\neproof
\hfill
\nsproof
\RightLabel{AssErr}
\AxiomC{$\Phi_1 \vdash$ t$_1$ ~$\stepsto$~ $\Phi_2 \vdash$ \emph{verror}}
\AxiomC{}
\BinaryInfC{$\Phi_1 \vdash$ t$_1$ \textbf{\tassig} t$_2$ ~$\stepsto$~ $\Phi_2$ $\vdash$ \emph{verror}}
\neproof

\caption{Heap related big-step semantics.}
\label{fig:heap-big}
\end{figure}



\section{Research contribution}
\label{sec:cont}
The language described in chapter \ref{sec:boolnat} has been implemented in Agda together with a small-step, big-step and denotational semantics.

\subsection{Properties}
\label{sec:metatheories}
The following meta-theories have been proved:
\begin{itemize}
	\item Progress
	\item Preservation
	\item Soundness
	\item Completeness
	\item Determinism
	\item Termination
	\item Type preservation in the heap
	\item Size preservation of the heap
\end{itemize}

Most of them are well-known properties of semantics. Those are proved mainly by case analyses and structural induction. By completeness and soundness follow that the big-step semantics is equivalent to the denotational semantics. At the moment the similar correspondence from small-step semantics and big-step semantics holds only in one direction. Nevertheless we expect it to hold as well, but because of the specific definitions used the proof is overly complicated, although not very interesting.

\paragraph{Type preservation in the heap}
This property states that when a cell of the heap is allocated, the type of the values that can be stored in that cell is fixed and cannot be changed. Although we did not use this property, we believe it could be useful in future work.

\paragraph{Size preservation of the heap}
This simple property states that the heap cannot shrink, but only grows.
In our language it is not possible to deallocate something from the heap.

\subsection{Hoare Logic}
\label{sec:hoare}
Hoare logic is a formal system used to reason about the correctness of a program. 
This is achieved using Hoare triples $\left\langle P \right\rangle   S \left\langle Q \right\rangle$, where $P$ and $Q$ are predicates over the state and $S$ is a program, which means that if $P$ holds in some initial state, and $P$ is evaluated leading to some other state, in this final state $Q$ holds.
We embedded this logic in our language and proved few inference rules so that it is actually possible to formally prove the correctness of a program written in our language.

\paragraph{Hoare Triples}
Hoare triples $ \left\langle P \right\rangle   S \left\langle Q \right\rangle$ have been modeled following the definition. The only differences is that the state in our  setting is the heap. In order to make triples more expressive the post condition is a predicate over not only the final heap, but also the value resulting from the evaluation of $S$.

\sproof
\AxiomC{$\vdash (P\ \Phi_0)$}
\AxiomC{$\Phi_0 \vdash S \Downarrow \Phi_1 \vdash v $}
\BinaryInfC{$\vdash (Q\ \Phi_1\ v)$}
\eproof
Where $\vdash (P\  \Phi_0)$ means that $P$ holds for the initial heap $\Phi_0$ and $\vdash (Q\ \Phi_1\ v)$ means that $Q$ holds for the final heap $\Phi_1$ and for the resulting value $v$.

Hoare triples come with two interpretations: total and partial.
When using partial interpretation the termination of the program is assumed whereas in total interpretation it is a proof obligation.
In our project we have provided either the interpretations, with the only difference that they refer to exceptions rather than non-termination.
The semantics of the language specifies that when an exception is raised the computation is aborted, therefore it is not possible to guarantee the post condition.
Partial interpretation is denoted by $ \left\langle P \right\rangle  S \left\langle Q \right\rangle^*$ :
\sproof
\AxiomC{$\vdash (P\ \Phi_0)$}
\AxiomC{$\Phi_0 \vdash S \Downarrow \Phi_1 \vdash v $}
\BinaryInfC{$\vdash \neg \texttt{fail} \Rightarrow (Q\ \Phi_1\ v)$}
\eproof
Where \texttt{fail} is true when $v$ is \texttt{verror}.

\section{Conclusion}
\label{sec:conclusion}
In this project we have designed and formalized a minimal language with mutable heap and exceptions. We have implemented it and its semantics using Agda and we have proved several meta-theories. 
Lastly we have embedded hoare logic  in the language, allowing to reason about the correctness of programs written in this language.

\subsection{Comparison}
The language we have studied in this project is more complex than the languages studied during the course because it is partial and the global state affects the evaluation of terms. In particular the operational semantics is not a relation between terms only but includes also the initial and final heap. In the same way the denotational semantics is a function from terms and heaps to values and heaps. It is crucial then to define the data types conveniently using indexes and parameters properly. In fact most of the problems we had during the project were caused ultimately by an inappropriate design of our data types, thus we suggest to always think carefully about these issues during the initial phase of this kind of projects.

The proofs of the meta-theories, apart from the presence of the heap, are not inherently more difficult than those seen during the course, because the structure of the proofs (structural induction and case analyses) is essentially the same.
In our case they are simply longer due to the greater number of constructs. Again sometimes we had to find proper definitions of our data types to express the theorems conveniently.

\subsection{Future Work}
Future work should be focused on the hoare logic part.
Firstly the set of rules provided with the project is not complete, thus other rules should be added in order to cover all the constructs of the language.

Secondly, although we proved a few hoare inference rules, from our experience it is rather difficult to use them directly to prove the correctness of a program.
The problem is that several implicit parameters need to be filled in manually, which makes the proof tedious and verbose.
It would be interesting to rewrite the rules so that they could be used more smoothly and effectively.

Lastly hoare logic should be extended to fully cope with statefull evaluation.
In our language there is no distinction between expressions and statements 
(an expression is a term that does not change the heap when evaluated).
However some hoare rules rely on the fact that the evaluation will not change the heap (see for example the If-Then-Else rule). It is
not possible to lift this restriction, because evaluating a term then might lead to a state in which the precondition is no more valid. Therefore with the available 
rules it is possible to reason about a limited set of programs.
A general workaround is to embed the sequence rule inside this kind of rules, however this leads to code duplication. We believe that this is a general pattern in our hoare rules, therefore it would be helpful to express this directly in the definition of hoare triples. 


\begin{thebibliography}{1}

\bibitem{BP02}
  Benjamin C. Pierce.
  \emph{Types and Programming Languages}

\bibitem{BPSF}
	Benjamin C. Pierce.
	\emph{Software Foundations and Programming Languages}


\end{thebibliography}



\begin{appendices}
The letter $\Phi$ will be used to denote heaps. The ":=" symbol is used here for assignment for better readability. In the
Agda source code assignment is represented by "\textless--".
\section{Small-step Semantics}
\label{app:small}

\subsection{If then else}

\nsproof
\RightLabel{If}
\AxiomC{\sstep[0][1]{t$_0$}{t$_0$'}}
\UnaryInfC{\sstep[0][1]{\textbf{if} t$_0$ \textbf{then} t$_1$ \textbf{else} t$_2$}{\textbf{if} t$_0$' \textbf{then} t$_1$ \textbf{else} t$_2$}}
\neproof
\\\\
\nsproof
\RightLabel{IfTrue}
\AxiomC{}
\UnaryInfC{\sstep{\textbf{if} \emph{true} \textbf{then} t \textbf{else} e}{t}}
\neproof
\hfill
\nsproof
\RightLabel{IfFalse}
\AxiomC{}
\UnaryInfC{\sstep{\textbf{if} \emph{false} \textbf{then} t \textbf{else} e}{e}}
\neproof
\\\\
\nsproof
\RightLabel{IfErr}
\AxiomC{}
\UnaryInfC{\sstep{\textbf{if} \emph{error} \textbf{then} t \textbf{else} e}{\emph{error}}}
\neproof


\subsection{Is zero?}

\nsproof
\RightLabel{IsZero}
\AxiomC{\sstep[0][1]{t}{t'}}
\UnaryInfC{\sstep[0][1]{\textbf{iszero} t}{\textbf{iszero} t'}}
\neproof
\hfill
\nsproof
\RightLabel{IsZeroZ}
\AxiomC{}
\UnaryInfC{\sstep{\textbf{iszero} (\emph{num} 0)}{\emph{true}}}
\neproof
\\\\
\nsproof
\RightLabel{IsZeroS}
\AxiomC{}
\UnaryInfC{\sstep{\textbf{iszero} (\emph{suc} v)}{\emph{false}}}
\neproof
\hfill
\nsproof
\RightLabel{IsZeroErr}
\AxiomC{}
\UnaryInfC{\sstep{\textbf{iszero} \emph{error}}{\emph{error}}}
\neproof

\subsection{References}

\nsproof
\RightLabel{New}
\AxiomC{\sstep[0][1]{t}{t'}}
\UnaryInfC{\sstep[0][1]{\textbf{new} t} {\textbf{new} t'}}
\neproof
\hfill
\nsproof
\RightLabel{NewVal}
\AxiomC{t $\equiv$ $\ulcorner$ v $\urcorner$}
\UnaryInfC{\sstepe{\Phi}{\Phi \doubleplus \text{[v]}}{\textbf{new} t}{\emph{ref} (\underline{length} $\Phi$)}}
\neproof
\\\\
\nsproof
\RightLabel{Deref}
\AxiomC{\sstep[0][1]{t}{t'}}
\UnaryInfC{\sstep[0][1]{\textbf{!} t} {\textbf{!} t'}}
\neproof
\hfill
\nsproof
\RightLabel{DerefVal}
\AxiomC{}
\UnaryInfC{\sstep{\textbf{!} (\emph{ref} n)} {$\ulcorner$ \underline{lookup} n $\Phi$ $\urcorner$}}
\neproof
\\\\
\nsproof
\RightLabel{DerefErr}
\AxiomC{}
\UnaryInfC{\sstep{\textbf{!} \emph{error}} {\emph{error}}}
\neproof
\hfill
\nsproof
\RightLabel{AssLeft}
\AxiomC{\sstep[0][1]{t$_1$}{t$_1$'}}
\UnaryInfC{\sstep[0][1]{t$_1$ \textbf{\tassig} t$_2$} {t$_1$' \textbf{\tassig} t$_2$}}
\neproof
\\\\
\nsproof
\RightLabel{AssRight}
\AxiomC{$\neg$\underline{isError} v ~~ \sstep[0][1]{t}{t'}}
\UnaryInfC{\sstep[0][1]{v \textbf{\tassig} t} {v \textbf{\tassig} t'}}
\neproof
\hfill
\nsproof
\RightLabel{AssErr}
\AxiomC{}
\UnaryInfC{\sstep{\emph{error} \textbf{\tassig} t} {\emph{error}}}
\neproof
\\\\
\nsproof
\RightLabel{AssRed}
\AxiomC{t $\equiv$ $\ulcorner$ v $\urcorner$ ~~ \textbf{Elem} $\Phi$ r}
\UnaryInfC{\sstepe{\Phi}{\text{\underline{replace}}~\Phi~\text{r v}}
{(\emph{ref} r) \textbf{\tassig} t}{t}}
\neproof
\hfill
\nsproof
\RightLabel{AssRedErr}
\AxiomC{$\neg$ \textbf{Elem} $\Phi$ r}
\UnaryInfC{\sstep{(\emph{ref} r) \textbf{\tassig} t} {\emph{error}}}
\neproof


\subsection{Sequencing}

\nsproof
\RightLabel{Seq}
\AxiomC{\sstep[0][1]{t$_1$}{t$_1$'}}
\UnaryInfC{\sstep[0][1]{t$_1$ \textbf{$\gg$} t$_2$} {t$_1$' \textbf{$\gg$} t$_2$}}
\neproof
\hfill
\nsproof
\RightLabel{SeqVal}
\AxiomC{$\neg$\underline{isError} v$_1$}
\UnaryInfC{\sstep{v$_1$ \textbf{$\gg$} t$_2$} {t$_2$}}
\neproof
\\\\
\nsproof
\RightLabel{SeqErr}
\AxiomC{}
\UnaryInfC{\sstep{\emph{error} \textbf{$\gg$} t} {\emph{error}}}
\neproof


\subsection{Try catch}

\nsproof
\RightLabel{TryCatch}
\AxiomC{\sstep[0][1]{t$_1$}{t$_1$'}}
\UnaryInfC{\sstep[0][1]{\textbf{try} t$_1$ \textbf{catch} t$_2$} {\textbf{try} t$_1$' \textbf{catch} t$_2$}}
\neproof
\hfill
\nsproof
\RightLabel{TryCatchSucc}
\AxiomC{$\neg$\underline{isError} v$_1$}
\UnaryInfC{\sstep{\textbf{try} v$_1$ \textbf{catch} t$_2$} {v$_1$}}
\neproof
\\\\
\nsproof
\RightLabel{TryCatchFail}
\AxiomC{\underline{isError} t$_1$}
\UnaryInfC{\sstep{\textbf{try} t$_1$ \textbf{catch} t$_2$} {t$_2$}}
\neproof


\section{Big-step Semantics}
\label{app:big}

\subsection{Values}
\nsproof
\RightLabel{True}
\AxiomC{}
\UnaryInfC{$\Phi \vdash$ \textbf{true} ~$\stepsto$~ $\Phi \vdash$ \emph{vtrue}}
\neproof
%
\nsproof
\RightLabel{False}
\AxiomC{}
\UnaryInfC{$\Phi \vdash$ \textbf{false} ~$\stepsto$~ $\Phi \vdash$ \emph{vfalse}}
\neproof
\hfill
\nsproof
\RightLabel{Num}
\AxiomC{}
\UnaryInfC{$\Phi \vdash$ \textbf{num} m ~$\stepsto$~ $\Phi \vdash$ \emph{vnum} m}
\neproof
\\\\
\nsproof
\RightLabel{Ref}
\AxiomC{}
\UnaryInfC{$\Phi \vdash$ \textbf{ref} r ~$\stepsto$~ $\Phi \vdash$ \emph{vref} r}
\neproof
\hfill
\nsproof
\RightLabel{Error}
\AxiomC{}
\UnaryInfC{$\Phi \vdash$ \textbf{error} ~$\stepsto$~ $\Phi \vdash$ \emph{verror}}
\neproof


\subsection{If then else}

\nsproof
\RightLabel{IfTrue}
\AxiomC{$\Phi_1 \vdash$ t$_0$ ~$\stepsto$~ $\Phi_2 \vdash$ \emph{vtrue}}
\AxiomC{$\Phi_2 \vdash$ t$_1$ ~$\stepsto$~ $\Phi_3 \vdash$ v}
\BinaryInfC{$\Phi_1 \vdash$ \textbf{if} t$_0$ \textbf{then} t$_1$ \textbf{else} t$_2$ ~$\stepsto$~ $\Phi_3 \vdash$ v}
\neproof
\hfill
\nsproof
\RightLabel{IfFalse}
\AxiomC{$\Phi_1 \vdash$ t$_0$ ~$\stepsto$~ $\Phi_2 \vdash$ \emph{vfalse}}
\AxiomC{$\Phi_2 \vdash$ t$_2$ ~$\stepsto$~ $\Phi_3 \vdash$ v}
\BinaryInfC{$\Phi_1 \vdash$ \textbf{if} t$_0$ \textbf{then} t$_1$ \textbf{else} t$_2$ ~$\stepsto$~ $\Phi_3 \vdash$ v}
\neproof
\\\\
\nsproof
\RightLabel{IfErr}
\AxiomC{$\Phi_1 \vdash$ t$_0$ ~$\stepsto$~ $\Phi_2 \vdash$ \emph{verror}}
\AxiomC{}
\BinaryInfC{$\Phi_1 \vdash$ \textbf{if} t$_0$ \textbf{then} t$_1$ \textbf{else} t$_2$ ~$\stepsto$~ $\Phi_2 \vdash$ \emph{verror}}
\neproof


\subsection{Is zero?}

\nsproof
\RightLabel{IsZeroZ}
\AxiomC{$\Phi_1 \vdash$ t ~$\stepsto$~ $\Phi_2 \vdash$ \emph{vnat 0}}
\AxiomC{}
\BinaryInfC{$\Phi_1 \vdash$ \textbf{isZero} t ~$\stepsto$~ $\Phi_2 \vdash$ \emph{vtrue}}
\neproof
\\\\
\nsproof
\RightLabel{IsZeroS}
\AxiomC{$\Phi_1 \vdash$ t ~$\stepsto$~ $\Phi_2 \vdash$ \emph{vnat (succ} n \emph{)}}
\AxiomC{}
\BinaryInfC{$\Phi_1 \vdash$ \textbf{isZero} t ~$\stepsto$~ $\Phi_2 \vdash$ \emph{vfalse}}
\neproof
\hfill
\nsproof
\RightLabel{IsZeroErr}
\AxiomC{$\Phi_1 \vdash$ t ~$\stepsto$~ $\Phi_2 \vdash$ \emph{verror}}
\AxiomC{}
\BinaryInfC{$\Phi_1 \vdash$ \textbf{isZero} t ~$\stepsto$~ $\Phi_2 \vdash$ \emph{verror}}
\neproof


\subsection{References}

\nsproof
\RightLabel{New}
\AxiomC{$\Phi_1 \vdash$ t ~$\stepsto$~ $\Phi_2 \vdash$ v}
\AxiomC{}
\BinaryInfC{$\Phi_1 \vdash$ \textbf{new} t ~$\stepsto$~ $\Phi_2 \doubleplus$ [v] $\vdash$ \emph{vref} (\underline{length} $\Phi_2$)}
\neproof
\hfill
\nsproof
\RightLabel{Deref}
\AxiomC{$\Phi_1 \vdash$ t ~$\stepsto$~ $\Phi_2 \vdash$ \emph{vref} r}
\AxiomC{}
\BinaryInfC{$\Phi_1 \vdash$ \textbf{!} t ~$\stepsto$~ $\Phi_2 \vdash$ \underline{lookup} r $\Phi_2$}
\neproof
\\\\
\nsproof
\RightLabel{DerefErr}
\AxiomC{$\Phi_1 \vdash$ t ~$\stepsto$~ $\Phi_2 \vdash$ \emph{verror}}
\AxiomC{}
\BinaryInfC{$\Phi_1 \vdash$ \textbf{!} t ~$\stepsto$~ $\Phi_2 \vdash$ \emph{verror}}
\neproof
\hfill
\nsproof
\RightLabel{Ass}
\AxiomC{\textbf{Elem} $\Phi_3$ r ~~ $\Phi_1 \vdash$ t$_1$ ~$\stepsto$~ $\Phi_2 \vdash$ \emph{vref} r}
\AxiomC{$\Phi_2 \vdash$ t$_2$ ~$\stepsto$~ $\Phi_3 \vdash$ v}
\BinaryInfC{$\Phi_1 \vdash$ t$_1$ \textbf{\tassig} t$_2$ ~$\stepsto$~ \underline{replace} $\Phi_3$ r v $\vdash$ v}
\neproof
\\\\
\nsproof
\RightLabel{AssOob}
\AxiomC{$\neg$ \textbf{Elem} $\Phi_3$ r}
\AxiomC{$\Phi_1 \vdash$ t$_1$ ~$\stepsto$~ $\Phi_2 \vdash$ \emph{vref} r}
\BinaryInfC{$\Phi_1 \vdash$ t$_1$ \textbf{\tassig} t$_2$ ~$\stepsto$~ $\Phi_2$ $\vdash$ \emph{verror}}
\neproof
\hfill
\nsproof
\RightLabel{AssErr}
\AxiomC{$\Phi_1 \vdash$ t$_1$ ~$\stepsto$~ $\Phi_2 \vdash$ \emph{verror}}
\AxiomC{}
\BinaryInfC{$\Phi_1 \vdash$ t$_1$ \textbf{\tassig} t$_2$ ~$\stepsto$~ $\Phi_2$ $\vdash$ \emph{verror}}
\neproof


\subsection{Sequencing}

\nsproof
\RightLabel{Seq}
\AxiomC{$\neg$ \textbf{isVError} v$_1$ ~~ $\Phi_1 \vdash$ t$_1$ ~$\stepsto$~ $\Phi_2 \vdash$ v$_1$}
\AxiomC{$\Phi_2 \vdash$ t$_2$ ~$\stepsto$~ $\Phi_3 \vdash$ v$_2$}
\BinaryInfC{$\Phi_1 \vdash$ t$_1$ \textbf{$\gg$} t$_2$ ~$\stepsto$~ $\Phi_3$ $\vdash$ v$_2$}
\neproof
\\\\
\nsproof
\RightLabel{SeqErr}
\AxiomC{$\Phi_1 \vdash$ t$_1$ ~$\stepsto$~ $\Phi_2 \vdash$ \emph{verror}}
\AxiomC{}
\BinaryInfC{$\Phi_1 \vdash$ t$_1$ \textbf{$\gg$} t$_2$ ~$\stepsto$~ $\Phi_3$ $\vdash$ \emph{verror}}
\neproof


\subsection{Try catch}

\nsproof
\RightLabel{TryCatch}
\AxiomC{$\neg$ \textbf{isVError} v$_1$}
\AxiomC{$\Phi_1 \vdash$ t$_1$ ~$\stepsto$~ $\Phi_2 \vdash$ v}
\BinaryInfC{$\Phi_1 \vdash$ \textbf{try} t$_1$ \textbf{catch} t$_2$ ~$\stepsto$~ $\Phi_2$ $\vdash$ v}
\neproof
\\\\
\nsproof
\RightLabel{TryCatchErr}
\AxiomC{$\Phi_1 \vdash$ t$_1$ ~$\stepsto$~ $\Phi_2 \vdash$ \emph{verror}}
\AxiomC{$\Phi_2 \vdash$ t$_2$ ~$\stepsto$~ $\Phi_3 \vdash$ v}
\BinaryInfC{$\Phi_1 \vdash$ \textbf{try} t$_1$ \textbf{catch} t$_2$ ~$\stepsto$~ $\Phi_3$ $\vdash$ v}
\neproof






\section{Hoare Rules}

\nsproof
\RightLabel{Precondition strengthening}
\AxiomC{$\models P \Rightarrow P'$}
\AxiomC{\hoare{P'}{S}{Q}}
\BinaryInfC{\hoare{P}{S}{Q}}
\neproof
\\\\
\nsproof
\RightLabel{Post condition weakening}
\AxiomC{$\models Q' \Rightarrow Q$}
\AxiomC{\hoare{P}{S}{Q'}}
\BinaryInfC{\hoare{P}{S}{Q}}
\neproof
\\\\
\nsproof
\RightLabel{Conjunction}
\AxiomC{\hoare{P1}{S}{Q1}}
\AxiomC{\hoare{P2}{S}{Q2}}
\BinaryInfC{\hoare{P1 \wedge P2}{S}{Q1 \wedge Q2}}
\neproof
\\\\
\nsproof
\RightLabel{Disjunction}
\AxiomC{\hoare{P1}{S}{Q1}}
\AxiomC{\hoare{P2}{S}{Q2}}
\BinaryInfC{\hoare{P1 \vee P2}{S}{Q1 \vee Q2}}
\neproof
\\\\
\nsproof
\RightLabel{If-Then-Else ($S$ expression)}
\AxiomC{\hoare{P \wedge g}{S1}{Q}}
\AxiomC{\hoare{P \wedge \neg g}{S2}{Q}}
\BinaryInfC{\hoare{P}{\texttt{if}\ g\ \texttt{then}\ S1\ \texttt{else}\ S2}{Q}}
\neproof
\\\\
\nsproof
\RightLabel{Sequencing}
\AxiomC{\hoare{P}{S1}{R}}
\AxiomC{\hoare{R}{S}{Q}}
\BinaryInfC{\hoare{P}{S1\ ;\ S2}{Q}}
\neproof
\\\\
\nsproof
\RightLabel{New}
\AxiomC{$\models P \Rightarrow Q_S$}
\UnaryInfC{\hoare{P}{\texttt{new}\ S}{Q}}
\neproof
\\\\
where $Q_S$ means that $Q$ holds for the heap in which the value evaluated from $S$ has been allocated.


\end{appendices}


\end{document}
