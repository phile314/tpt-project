\documentclass[12pt, a4paper, oneside]{article}
\usepackage{graphicx}
\usepackage{hyperref}
\usepackage{listings}
\usepackage{placeins}
\usepackage{amsmath}
\usepackage{amsfonts}
\usepackage{bussproofs}
\usepackage{xparse}

\usepackage{proof}

\newcommand\impfnt[1]{\mbox{\bf #1}}
\newcommand{\SKIP}{\impfnt{skip}}
\newcommand{\stepsto}{\Downarrow}
\newcommand{\Rule}[2]{\infer{#2}{#1}}
\newcommand{\RuleSide}[3]{\infer[#3]{#2}{#1}}

% font size could be 10pt (default), 11pt or 12 pt
% paper size coulde be letterpaper (default), legalpaper, executivepaper,
% a4paper, a5paper or b5paper
% side coulde be oneside (default) or twoside 
% columns coulde be onecolumn (default) or twocolumn
% graphics coulde be final (default) or draft 
%
% titlepage coulde be notitlepage (default) or titlepage which 
% makes an extra page for title 
% 
% paper alignment coulde be portrait (default) or landscape 
%
% equations coulde be 
%   default number of the equation on the rigth and equation centered 
%   leqno number on the left and equation centered 
%   fleqn number on the rigth and  equation on the left side
%   

\newcommand\doubleplus{+\kern-1.3ex+\kern0.8ex}
\newcommand\mdoubleplus{\ensuremath{\mathbin{+\mkern-10mu+}}}

\newcommand{\sproof}{
  \scriptsize
  \begin{center}
  \begin{prooftree}
  \def\defaultHypSeparation{\hskip .1in}
% whats that doing? ->  \def\fCenter{\models}
}

\newcommand{\eproof}{
  \end{prooftree}
  \end{center}
  \normalsize
}

\DeclareDocumentCommand{\sstepe}{m m m m} {%
  \ensuremath{{#1} \vdash \text{{#3}} \; \rightarrow \; {#2} \vdash \text{{#4}}}%
}
\DeclareDocumentCommand{\sstep}{O{} O{} m m} {%
  \sstepe{\Phi_{#1}}{\Phi_{#2}}{#3}{#4}%
}



\title{TPT Project (TODO add desc)}
\author{Marco Vassena  \\
    4110161 \\
    \and
    Philipp Hausmann \\
    4003373 \\
    \and
    Ondra Pelech \\
    F131636 \\
    }

\date{\today}
\begin{document}



\maketitle

\tableofcontents

blablalb

\section{Semantics}

\sproof
\AxiomC{\sstep[0][1]{t}{t'}}
\UnaryInfC{\sstep[0][1]{iszero t}{iszero t'}}
\eproof

\sproof
\AxiomC{}
\UnaryInfC{\sstep{iszero 0}{true}}
\eproof

\sproof
\AxiomC{}
\UnaryInfC{\sstep{iszero (suc v)}{false}}
\eproof

\sproof
\AxiomC{\sstep[0][1]{c}{c'}}
\UnaryInfC{\sstep[0][1]{if c then t else e}{if c' then t else e}}
\eproof

\sproof
\AxiomC{}
\UnaryInfC{\sstep{if true then t else e}{t}}
\eproof

\sproof
\AxiomC{}
\UnaryInfC{\sstep{if false then t else e}{e}}
\eproof

\sproof
\AxiomC{\sstep[0][1]{t}{t'}}
\UnaryInfC{\sstep[0][1]{new t} {new t'}}
\eproof

\sproof
\AxiomC{}
\UnaryInfC{\sstepe{\Phi}{\Phi \doubleplus [v]}{new v}{vref $len_\Phi$}}
\eproof

\section{Big-step Semantics}

\begin{itemize}

\item
New:
\begin{tabular}{c}
\infer{true \stepsto vtrue}{}
\end{tabular}

\item
IfTrue:
\begin{tabular}{c}
\infer{if~t~then~t1~else~t2 \stepsto v}{t \stepsto vtrue~~t1 \stepsto v}
\end{tabular}

\end{itemize}


\begin{thebibliography}{1}

\bibitem{PV14}
  Wishnu Prasetya.
  \emph{Lecture Notes Program Verification, v2014.0}.

\end{thebibliography}


\end{document}
